Phil Whiting received his BA degree from the University of Oxford, his
MSc from the University of London and his Ph. D. was in queueing
theory from the University of Strathclyde. After a post-doc at the
University of Cambridge, Phil's interests centered on wireless networks and
with Stephen Hanly pioneered the application of Information theory
to mobile wireless networks.

In 1993 Phil was a consultant at Telstra research labs during the trial of Qualcomm
CDMA in Melbourne. Subsequently he became Deputy Director for the Mobile Research Centre at the
Institute of Telecommunications Research.

From January 1997 to June 2013, Phil worked in the Maths Centre engaging in a variety of pure
and applied research topics including, multi-carrier wireless networks, Proportional Fair scheduling,
DSL vectoring, Large deviations for occupancy models and the theory of random matrices amongst others.

June 2013, Phil became a research fellow at MacQuarie University and also holds a position
as Visiting Scientist at CSIRO.


Phil has over 30 patents for Telecommunication networks in particular in
the areas of wireless, DSL and location techniques.  Phil was the lead scientist for the demonstration of
vectored DSL to Telecom New Zealand in 2009. Phil has won numerous awards and has been a vsiting scholar/Professor at various instituitons including the University of Korea, Seoul, Brown University, Rhode Island, CWI Amsterdam, Eindhoven University and Melbourne University.

Phil's current interests remain in the area of wireless networks including backlog based stabilising algorithms for CSMA and Heterogeneous Networks.
